% ============================================================================
% PRESENTACIÓ TFG
% Cristhian Omar Añez López - UAB 2025/26
% ============================================================================
\documentclass[aspectratio=169,10pt]{beamer}

% ============================================================================
% TEMA - Metropolis 
% ============================================================================
\usetheme{metropolis}
\usepackage[utf8]{inputenc}
\usepackage[catalan]{babel}
\usepackage{graphicx}
\usepackage{booktabs}
\usepackage{amsmath,amssymb}
\usepackage{tikz}
\usepackage{fontawesome5}
\usepackage{xcolor}
\usepackage{multicol}
\usepackage{appendixnumberbeamer}

% ============================================================================
% PALETA DE COLORS - Tema Espacial/Tecnològic
% ============================================================================
\definecolor{cosmicBlue}{RGB}{15, 32, 65}        % Blau fosc espacial
\definecolor{nebulaBlue}{RGB}{30, 90, 150}       % Blau nebulosa
\definecolor{starLight}{RGB}{255, 215, 100}      % Llum estel·lar
\definecolor{techCyan}{RGB}{0, 200, 220}         % Cian tecnològic
\definecolor{alertRed}{RGB}{220, 80, 80}         % Vermell alerta
\definecolor{successGreen}{RGB}{80, 180, 100}    % Verd èxit

% Configuració del tema amb colors personalitzats
\setbeamercolor{frametitle}{bg=cosmicBlue, fg=white}
\setbeamercolor{title}{fg=cosmicBlue}
\setbeamercolor{progress bar}{fg=techCyan, bg=cosmicBlue!30}
\setbeamercolor{block title}{bg=nebulaBlue, fg=white}
\setbeamercolor{block body}{bg=cosmicBlue!10}
\setbeamercolor{alerted text}{fg=alertRed}
\setbeamercolor{example text}{fg=successGreen}

% ============================================================================
% METADADES
% ============================================================================
\title{\textbf{Implementació Compressor Basat en Xarxes Neurals per a Ús en Satèl·lit}}
\subtitle{Portant Deep Learning a l'Espai amb Recursos Limitats}
\author{Cristhian Omar Añez López}
\institute{
    Escola d'Enginyeria (EE) \\
    Universitat Autònoma de Barcelona (UAB) \\[0.3em]
    \footnotesize{Tutor: Sebastià Mijares Verdú}
}
\date{Febrer 2026}

% ============================================================================
% DOCUMENT
% ============================================================================
\begin{document}

% ----------------------------------------------------------------------------
% PORTADA
% ----------------------------------------------------------------------------
{
\setbeamertemplate{footline}{}
\begin{frame}[plain]
    \maketitle
    \begin{center}
        \vspace{-1em}
        \footnotesize{\textit{TFG en Enginyeria Informàtica — Menció: Tecnologies de la Informació}}
    \end{center}
\end{frame}
}

% ----------------------------------------------------------------------------
% ÍNDEX
% ----------------------------------------------------------------------------
\begin{frame}{Contingut de la Presentació}
    \tableofcontents
\end{frame}

% ============================================================================
% SECCIÓ 1: INTRODUCCIÓ I MOTIVACIÓ
% ============================================================================
\section{Introducció i Motivació}

\begin{frame}{El Problema: Massa Dades, Poc Ample de Banda}
    \begin{columns}[T]
        \begin{column}{0.55\textwidth}
            \textbf{Context actual de l'observació terrestre:}
            \begin{itemize}
                \item Missions com \textbf{Sentinel-2}, EnMAP, PRISMA generen terabytes diaris
                \item Imatges multiespectrals/hiperespectrals d'alta resolució
                \item \alert{L'ample de banda de descàrrega és limitat}
            \end{itemize}
            
            \vspace{1em}
            \textbf{En CubeSats i nanosatèl·lits:}
            \begin{itemize}
                \item Potència molt limitada (watts)
                \item Poca memòria RAM ($\leq$ 1 GB)
                \item CPU de baix consum (ARM)
            \end{itemize}
            
            \vspace{1em}
            {\large \faArrowRight} \textbf{Cal comprimir les dades a bord!}
        \end{column}
        \begin{column}{0.4\textwidth}
            \vspace{4em}
            \begin{block}{El Desafiament}
                \centering
                \vspace{0.5em}
                \textit{Com executar algorismes intel·ligents de compressió en maquinari molt limitat?}
                \vspace{0.5em}
            \end{block}
            \vspace{1em}
            %\includegraphics[width=\textwidth]{img/SORTENY_DIAGRAM.png}
        \end{column}
    \end{columns}
\end{frame}

\begin{frame}{Objectius del TFG}
    \begin{block}{Objectiu General}
        Desenvolupar i validar una versió eficient del compressor \textbf{SORTENY} orientada a plataformes amb recursos limitats.
    \end{block}
    
    \vspace{1em}
    
    \begin{columns}[T]
        \begin{column}{0.48\textwidth}
            \textbf{\faCode\ Implementació}
            \begin{itemize}
                \item Reimplementar SORTENY en \textbf{C pur}
                \item Eliminar dependències de TensorFlow
                \item Arquitectura portable per ARM
            \end{itemize}
            
            \vspace{0.5em}
            \textbf{\faCogs\ Optimització}
            \begin{itemize}
                \item Reduir consum de memòria
                \item Explotar paral·lelisme (OpenMP)
                \item Vectorització SIMD (NEON)
            \end{itemize}
        \end{column}
        \begin{column}{0.48\textwidth}
            \textbf{\faChartLine\ Avaluació}
            \begin{itemize}
                \item Comparar amb Python/TensorFlow
                \item Benchmark vs CCSDS-122
                \item Validar qualitat de reconstrucció
            \end{itemize}
            
            \vspace{0.5em}
            \textbf{\faRocket\ Viabilitat}
            \begin{itemize}
                \item Provar en Raspberry Pi 3B+
                \item Simular restriccions de CubeSat
                \item Demostrar factibilitat real
            \end{itemize}
        \end{column}
    \end{columns}
\end{frame}

% ============================================================================
% SECCIÓ 2: ESTAT DE L'ART
% ============================================================================
\section{Estat de l'Art}

\begin{frame}{Compressió Espacial: Mètodes Clàssics (CCSDS)}
    \begin{columns}[T]
        \begin{column}{0.55\textwidth}
            \textbf{CCSDS 122.0-B-1: L'Estàndard Industrial}
            \begin{enumerate}
                \item \textbf{Transformada Wavelet (DWT)}
                    \begin{itemize}
                        \item Descomposició en sub-bandes
                        \item Filtres 5/3 (lossless) o 9/7 (lossy)
                        \item Compactació d'energia
                    \end{itemize}
                \item \textbf{Codificador de Plans de Bits (BPE)}
                    \begin{itemize}
                        \item Codificació progressiva MSB→LSB
                        \item Estructura d'arbres (Zerotrees)
                        \item Flux escalable
                    \end{itemize}
            \end{enumerate}
            
            \vspace{0.5em}
            \textcolor{successGreen}{\faCheck} Ràpid, eficient, provat en vol \\
            \textcolor{alertRed}{\faTimes} Transformades \textbf{fixes}, no adaptatives
        \end{column}
        \begin{column}{0.42\textwidth}
            \includegraphics[width=\textwidth]{img/ccsds_schematic.png}
            \vspace{0.5em}
            \includegraphics[width=\textwidth]{img/ccsds_dwt_3lvl.png}
        \end{column}
    \end{columns}
\end{frame}

\begin{frame}{Compressió Apresa: El Model SORTENY}
    \begin{columns}[T]
        \begin{column}{0.45\textwidth}
            \textbf{SORTENY: Spectral Orthogonal Transform Encoder}
            
            \vspace{0.5em}
            Desenvolupat per l'IEEC, combina:
            \begin{enumerate}
                \item \textbf{Transformada Espectral Apresa}
                    \begin{itemize}
                        \item Similar a PCA/KLT
                        \item Descorrelació de bandes
                    \end{itemize}
                \item \textbf{CNN per Compressió Espacial}
                    \begin{itemize}
                        \item Convolucions amb stride
                        \item Normalització GDN
                    \end{itemize}
                \item \textbf{Modulació de Qualitat ($\lambda$)}
                    \begin{itemize}
                        \item Taxa variable sense reentrenar
                        \item Un únic model per tot
                    \end{itemize}
            \end{enumerate}
        \end{column}
        \begin{column}{0.52\textwidth}
            \begin{center}
                \includegraphics[width=17em]{img/SORTENY_DIAGRAM.png}
            \end{center}
            %%\vspace{0.3em}
            \begin{block}{Avantatge Clau}
                \textbf{Adaptabilitat}: Les transformades s'aprenen de les dades reals, optimitzant qualitat per bit.
            \end{block}
        \end{column}
    \end{columns}
\end{frame}

\begin{frame}{El Problema de SORTENY}
    \begin{columns}[T]
        \begin{column}{0.48\textwidth}
            \begin{alertblock}{Requisits de la Versió Python/TensorFlow}
                \begin{itemize}
                    \item \textbf{RAM:} 800 MB només per carregar el model
                    \item \textbf{Dependències:} TensorFlow, NumPy, Python runtime
                    \item \textbf{Pes:} Gigabytes d'instal·lació
                \end{itemize}
            \end{alertblock}
            
            \vspace{0.5em}
            \textbf{En una Raspberry Pi 3B+:}
            \begin{itemize}
                \item 1 GB RAM total (compartida amb GPU)
                \item El sistema col·lapsa (OOM Kill)
                \item Swap massiu → rendiment catastròfic
            \end{itemize}
        \end{column}
        \begin{column}{0.48\textwidth}
            \begin{exampleblock}{La Pregunta Clau}
                \centering
                \vspace{0.5em}
                \Large{\textit{És viable executar un compressor basat en xarxes neuronals en un entorn embarcat realista?}}
                \vspace{0.5em}
            \end{exampleblock}
            
            \vspace{1em}
            \begin{center}
                {\Huge \faArrowDown}
                
                \vspace{0.5em}
                \textbf{Sí, però com?}
            \end{center}
        \end{column}
    \end{columns}
\end{frame}

% ============================================================================
% SECCIÓ 3: METODOLOGIA
% ============================================================================
\section{Metodologia i Implementació}

\begin{frame}{Estratègia: Reimplementació Nativa en C}
    \begin{columns}[T]
        \begin{column}{0.48\textwidth}
            \textbf{Fase 1: Extracció de Pesos}
            \begin{itemize}
                \item Serialitzar tensors a binaris plans
                \item Generar índex de metadades
            \end{itemize}
            
            \vspace{0.5em}
            \textbf{Fase 2: Motor d'Inferència}
            \begin{itemize}
                \item \textbf{Convolució 2D:} Padding SAME manual
                \item \textbf{GDN:} Normalització divisiva
                \item \textbf{Transformada Espectral:} Producte matricial
            \end{itemize}
            
            \vspace{0.5em}
            \textbf{Resultat:}
            \begin{itemize}
                \item \textcolor{successGreen}{\faCheck} Zero dependències
                \item \textcolor{successGreen}{\faCheck} Executable únic i portable
            \end{itemize}
        \end{column}
        \begin{column}{0.48\textwidth}
            \begin{block}{Desacoblament del Framework}
                \begin{center}
                    \begin{tikzpicture}[scale=0.8]
                        % Python box
                        \node[draw, fill=alertRed!20, minimum width=3cm, minimum height=1.2cm, rounded corners] at (0,2) {\textbf{Python + TF}};
                        \node at (0,1.5) {\footnotesize 800 MB RAM};
                        
                        % Arrow
                        \node at (0,0.7) {\Large\faArrowDown};
                        
                        % C box
                        \node[draw, fill=successGreen!20, minimum width=3cm, minimum height=1.2cm, rounded corners] at (0,-0.5) {\textbf{C11 Natiu}};
                        \node at (0,-1.0) {\footnotesize ?? MB RAM};
                    \end{tikzpicture}
                \end{center}
            \end{block}
            
            \vspace{0.5em}
            \begin{block}{Compilació Optimitzada}
                \footnotesize
                \texttt{gcc -O3 -mcpu=cortex-a53 \\
                -mfpu=neon-vfpv4 -fopenmp}
            \end{block}
        \end{column}
    \end{columns}
\end{frame}

\begin{frame}{Gestió de Memòria: Estratègia ``Ping-Pong''}
    \begin{columns}[T]
        \begin{column}{0.55\textwidth}
            \textbf{Problema Original:}
            \begin{itemize}
                \item Cada capa crea nous tensors
                \item Memòria creix linealment $O(N)$
                \item Pics de 800+ MB
            \end{itemize}
            
            \vspace{0.5em}
            \textbf{Solució Ping-Pong:}
            \begin{itemize}
                \item Només \textbf{2 buffers} de treball
                \item Alternança: $A \rightarrow B \rightarrow A \rightarrow B$
                \item Memòria constant $O(1)$
            \end{itemize}
            
            \vspace{0.5em}
            \textbf{Fórmula de consum:}
            \begin{equation*}
                \text{Mem}_{\text{total}} \approx \text{Pesos} + \max_{i}(\text{Dim Capa}_i) \times 2
            \end{equation*}
        \end{column}
        \begin{column}{0.42\textwidth}
            \begin{center}
                \vspace{3em}
                \begin{tikzpicture}[scale=1.0]
                    % Buffers
                    \node[draw, fill=techCyan!30, minimum width=2.5cm, minimum height=0.8cm] (A) at (0,3) {Buffer A};
                    \node[draw, fill=starLight!50, minimum width=2.5cm, minimum height=0.8cm] (B) at (0,1.5) {Buffer B};
                    
                    % Layers
                    \node[draw, fill=cosmicBlue!20, circle, minimum size=0.8cm] (L1) at (-2.5,2.25) {$L_1$};
                    \node[draw, fill=cosmicBlue!20, circle, minimum size=0.8cm] (L2) at (2.5,2.25) {$L_2$};
                    
                    % Arrows
                    \draw[->, thick, techCyan] (L1) -- (A);
                    \draw[->, thick, techCyan] (B) -- (L1);
                    \draw[->, thick, starLight!80!black] (A) -- (L2);
                    \draw[->, thick, starLight!80!black] (L2) -- (B);
                    
                    % Label
                    \node at (0,0.3) {\footnotesize Reutilització cíclica};
                \end{tikzpicture}
            \end{center}
            
            %\vspace{0.5em}
            %\begin{exampleblock}{Resultat}
            %    \centering
            %    Reducció de memòria:\\
            %    \textbf{800 MB → 87 MB}\\
            %    (\textbf{9.2x} menys)
            %\end{exampleblock}
        \end{column}
    \end{columns}
\end{frame}

\begin{frame}{Optimització: Paral·lelisme i Vectorització}
    \begin{columns}[T]
        \begin{column}{0.48\textwidth}
            \begin{block}{\faLayerGroup\ Paral·lelisme de Fils (OpenMP)}
                \begin{itemize}
                    \item Independència en canals de sortida
                    \item Distribució en 4 nuclis Cortex-A53
                    \item Directiva: \texttt{\#pragma omp parallel for}
                    %\item Ús CPU: \textbf{338\%} (satura els 4 nuclis)
                \end{itemize}
            \end{block}
        \end{column}
        \begin{column}{0.48\textwidth}
            \begin{block}{\faMicrochip\ Vectorització SIMD (NEON)}
                \begin{itemize}
                    %\item Unitats vectorials de 128 bits
                    \item Processar múltiples píxels per cicle
                    \item Flags: \texttt{-mcpu=cortex-a53 -mfpu=neon}
                    \item Compilació: \texttt{-O3 -funsafe-math}
                \end{itemize}
            \end{block}
        \end{column}
    \end{columns}
\end{frame}

% ============================================================================
% SECCIÓ 4: RESULTATS EXPERIMENTALS
% ============================================================================
\section{Resultats Experimentals}

\begin{frame}{Configuració de l'Experiment}
    \begin{columns}[T]
        \begin{column}{0.48\textwidth}
            \textbf{\faServer\ Plataforma de Proves}
            \begin{itemize}
                \item \textbf{Hardware:} Raspberry Pi 3B+
                \item \textbf{CPU:} ARM Cortex-A53 @ 1.4 GHz
                \item \textbf{RAM:} 1 GB LPDDR2
                \item \textbf{OS:} Raspberry Pi OS 64-bit
            \end{itemize}
            
            \vspace{0.5em}
            \textbf{\faImage\ Dataset}
            \begin{itemize}
                \item Imatge Sentinel-2 (T31TCG)
                \item Dimensions: $512 \times 512$ px
                \item 8 bandes espectrals
                \item 16 bits/píxel (4.19 MB total)
            \end{itemize}
        \end{column}
        \begin{column}{0.48\textwidth}
            \textbf{\faFlask\ Comparatives}
            \begin{table}
                \footnotesize
                \begin{tabular}{ll}
                    \toprule
                    \textbf{Mètode} & \textbf{Detalls} \\
                    \midrule
                    SORTENY C & GCC 10.2, OpenMP \\
                    SORTENY Py & Python 3.9, TF 2.16 \\
                    CCSDS 122 & MHDC (UAB) \\
                    \bottomrule
                \end{tabular}
            \end{table}
            
            \vspace{0.5em}
            \begin{block}{Paràmetre de Qualitat}
                \centering
                $\lambda = 0.1$ \\
                (Alta qualitat)
            \end{block}
        \end{column}
    \end{columns}
\end{frame}

\begin{frame}{Resultats: Rendiment i Memòria}
    \begin{columns}[T]
        \begin{column}{0.55\textwidth}
            \begin{center}
                \includegraphics[width=\textwidth, height=0.7\textheight]{img/fig5_rendimiento.png}
            \end{center}
        \end{column}
        \begin{column}{0.42\textwidth}
            \textbf{Anàlisi:}
            \begin{itemize}
                \item \textbf{C:} Satura els 4 nuclis (338\% CPU)
                \item \textbf{Python:} 75\% del temps en swap/IO
                \item La memòria era el \alert{coll d'ampolla real}
            \end{itemize}

            \vspace{1em}
            \begin{exampleblock}{Xifres Clau}
                \textbf{Speedup:} 1.47x\\  
                \textbf{RAM:} 796 → 87 MB (9.2x)\\  
                \textbf{CPU:} 338\% vs 27\%
            \end{exampleblock}
        \end{column}
    \end{columns}
\end{frame}

\begin{frame}{Resultats: Qualitat de Reconstrucció}
    \begin{columns}[T]
        \begin{column}{0.55\textwidth}
            \begin{center}
                \includegraphics[width=\textwidth, height=0.7\textheight]{img/fig4_metricas_calidad.png}
            \end{center}
        \end{column}
        \begin{column}{0.42\textwidth}
            \textbf{Anàlisi:}
            \begin{itemize}
                \item Les dues línies (C i Python) se superposen perfectament
                \item Variació natural per contingut espectral
                \item Diferències $<$ 0.03 dB
            \end{itemize}

            \vspace{1em}
            \begin{exampleblock}{Conclusió}
                \textbf{PSNR Global: 76.73 dB}\\  
                Qualitat \textbf{idèntica}\\  
                entre C i Python
            \end{exampleblock}
        \end{column}
    \end{columns}
\end{frame}

\begin{frame}{Validació Visual i Paritat de Latents}
    \begin{columns}[T]
        \begin{column}{0.48\textwidth}
            \textbf{Paritat Numèrica:}
            \begin{itemize}
                \item 3.15 milions de valors latents
                \item \textbf{99.9876\%} són idèntics
                \item Només 390 amb diferència $\pm 1$
            \end{itemize}
            
            \vspace{0.5em}
            \textbf{Causa de les diferències:}
            \begin{itemize}
                \item Precisió FP: NEON vs AVX/SSE
                \item Arrodoniment bancari en $x.5$
            \end{itemize}
            
            \vspace{0.5em}
            \begin{center}
                \includegraphics[width=\textwidth, height=0.4\textheight]{img/fig6_histograma_latentes.png}
            \end{center}
        \end{column}
        \begin{column}{0.48\textwidth}
            \textbf{Inspecció Visual:}
            \begin{center}
                \includegraphics[width=\textwidth, height=0.65\textheight]{img/fig8_detalle_banda1.png}
            \end{center}
            \footnotesize{Detall Banda 1: Original vs C vs Python}
            
            \vspace{0.3em}
            {\footnotesize Imatges \textbf{visualment indistingibles}}
        \end{column}
    \end{columns}
\end{frame}

\begin{frame}{Comparativa amb CCSDS-122 (Benchmark Industrial)}
    \begin{table}
        \caption{SORTENY ($\lambda=0.1$) vs CCSDS-122}
        \begin{tabular}{lccccc}
            \toprule
            \textbf{Mètode} & \textbf{Temps} & \textbf{RAM} & \textbf{Ratio} & \textbf{PSNR} & \textbf{MSE} \\
            \midrule
            CCSDS 122 (Lossless) & \textbf{12.6 s} & \textbf{25 MB} & 1.79:1 & $\infty$ & 0.00 \\
            CCSDS 122 (Near-LL) & 12.7 s & 25 MB & 1.79:1 & \textbf{103.5 dB} & \textbf{0.19} \\
            \midrule
            SORTENY C & 304 s & 87 MB & $\sim$2.5:1* & 76.7 dB & 91.3 \\
            \bottomrule
        \end{tabular}
        
        \footnotesize{*Ratio estimat basat en entropia (sense codificador aritmètic)}
    \end{table}
    
    \vspace{0.5em}
    
    \begin{columns}[T]
        \begin{column}{0.48\textwidth}
            \textcolor{successGreen}{\faCheck} \textbf{CCSDS guanya en:}
            \begin{itemize}
                \item Velocitat (24x més ràpid)
                \item Memòria (3x menys)
                \item Qualitat lossless perfecta
            \end{itemize}
        \end{column}
        \begin{column}{0.48\textwidth}
            \textcolor{techCyan}{\faCheck} \textbf{SORTENY guanya en:}
            \begin{itemize}
                \item Ratio de compressió potencial
                \item \textbf{Adaptabilitat} (reentrenable)
                \item Actualitzable en vol
            \end{itemize}
        \end{column}
    \end{columns}
\end{frame}

% ============================================================================
% SECCIÓ 5: CONCLUSIONS
% ============================================================================
\section{Conclusions}

\begin{frame}{Conclusions Principals}
    \begin{columns}[T]
        \begin{column}{0.48\textwidth}
            \begin{exampleblock}{\faCheckCircle\ Èxits Aconseguits}
                \begin{enumerate}
                    \item \textbf{Viabilitat Demostrada}
                        \begin{itemize}
                            \item RAM: 796 MB → 87 MB (\textbf{-90\%})
                            \item El sistema ja no col·lapsa
                        \end{itemize}
                    \item \textbf{Qualitat Preservada}
                        \begin{itemize}
                            \item PSNR idèntic (76.73 dB)
                            \item Paritat 99.98\%
                        \end{itemize}
                    \item \textbf{Rendiment Millorat}
                        \begin{itemize}
                            \item Speedup 1.47x
                            \item 4 nuclis al 338\%
                        \end{itemize}
                \end{enumerate}
            \end{exampleblock}
        \end{column}
        \begin{column}{0.48\textwidth}
            \begin{block}{\faLightbulb\ Aprenentatges Clau}
                \begin{itemize}
                    \item La \textbf{memòria} és la barrera principal, no la CPU
                    \item És possible portar DL a sistemes embarcats \textbf{amb enginyeria adequada}
                    %\item Les transformades apreses ofereixen \textbf{flexibilitat} vs fixes
                \end{itemize}
            \end{block}
            
            \vspace{0.5em}
            \begin{alertblock}{\faExclamationTriangle\ Limitacions}
                \begin{itemize}
                    \item Codificador entròpic pendent
                    \item CCSDS encara és més ràpid
                \end{itemize}
            \end{alertblock}
        \end{column}
    \end{columns}
\end{frame}

\begin{frame}{Treball Futur}
    \begin{columns}[T]
        \begin{column}{0.32\textwidth}
            \begin{block}{\faCode\ Codificador Entròpic}
                Integrar Montsec al mateix executable C per mesurar ràtios reals (bpp).
            \end{block}
        \end{column}
        \begin{column}{0.32\textwidth}
            \begin{block}{\faMicrochip\ Acceleració HW}
                Optimització manual amb intrínsecs NEON o Arm Compute Library.
            \end{block}
        \end{column}
        \begin{column}{0.32\textwidth}
            \begin{block}{\faRocket\ Validació en Òrbita}
                Portar a CubeSat real: buit tèrmic, radiació (SEUs), TRL elevat.
            \end{block}
        \end{column}
    \end{columns}
    
    \vspace{1em}
    
    %\begin{center}
    %    \begin{tikzpicture}
    %        \node[draw, fill=cosmicBlue!20, rounded corners, minimum width=10cm, minimum height=1cm] {
    %            \textbf{Visió:} Actualitzar xarxes neuronals en vol pujant només els pesos (KB)
    %        };
    %    \end{tikzpicture}
    %\end{center}
\end{frame}

% ----------------------------------------------------------------------------
% SLIDE FINAL
% ----------------------------------------------------------------------------
{
\setbeamertemplate{footline}{}
\begin{frame}[plain]
    \begin{center}
        \vspace{2cm}
        {\Huge \textbf{Gràcies per la vostra atenció!}}
        
        \vspace{1.5cm}
        {\Large Preguntes?}
        
        \vspace{1.5cm}
        \faEnvelope\ \texttt{1635157@uab.cat}
        
        \vspace{0.5cm}
        {\footnotesize Cristhian Omar Añez López — TFG Enginyeria Informàtica — UAB 2025/26}
    \end{center}
\end{frame}
}

% ============================================================================
% APÈNDIX (SLIDES EXTRA PER SI CAL)
% ============================================================================
\appendix

\begin{frame}[plain]
    \begin{center}
        {\Large \textbf{Slides Addicionals}}
    \end{center}
\end{frame}

\begin{frame}{Arquitectura Detallada SORTENY}
    \begin{columns}[T]
        \begin{column}{0.45\textwidth}
            \textbf{Etapes del Compressor:}
            \vspace{0.5em}
            \begin{enumerate}
                \item \textbf{(A) Transformada Espectral:}
                    \begin{itemize}
                        \item $x' = Ax + b$
                        \item Similar a PCA/KLT
                        \item Descorrelació de bandes
                    \end{itemize}
                \item \textbf{(B) CNN Espacial:}
                    \begin{itemize}
                        \item Conv2D amb stride
                        \item Normalització GDN
                        \item Downsampling progressiu
                    \end{itemize}
                \item \textbf{(C) Modulació:}
                    \begin{itemize}
                        \item $\hat{Y} = \text{round}(Y \cdot M(\lambda))$
                        \item Control de qualitat/taxa
                    \end{itemize}
            \end{enumerate}
        \end{column}
        \begin{column}{0.52\textwidth}
            \begin{center}
                \includegraphics[width=\textwidth]{img/SORTENY_DIAGRAM.png}
            \end{center}
        \end{column}
    \end{columns}
\end{frame}

\begin{frame}{Detall: Normalització GDN}
    \begin{columns}[T]
        \begin{column}{0.48\textwidth}
            \textbf{Fórmula Matemàtica:}
            \begin{equation*}
                y_i = \frac{x_i}{\sqrt{\beta_i + \sum_j \gamma_{ij} x_j^2}}
            \end{equation*}
            
            \vspace{0.5em}
            \textbf{Per què GDN i no ReLU?}
            \begin{itemize}
                \item ``Gaussianitza'' les dades
                \item Coeficients més independents
                \item Millor per a codificació entròpica
            \end{itemize}
        \end{column}
        \begin{column}{0.48\textwidth}
            \textbf{Implementació en C:}
            \begin{itemize}
                \item Paràmetres $\beta_i$, $\gamma_{ij}$ apresos
                \item Gestió de $\epsilon$ per estabilitat
                \item Bucles paral·lelitzats amb OpenMP
            \end{itemize}
        \end{column}
    \end{columns}
\end{frame}

\begin{frame}{Comparativa Visual Completa}
    \begin{center}
        \includegraphics[width=0.9\textwidth]{img/fig1_bandas_comparacion.png}
    \end{center}
    \footnotesize{Dalt: Original | Mig: SORTENY C | Baix: SORTENY Python}
\end{frame}

\begin{frame}{Mapes d'Error}
    \begin{center}
        \includegraphics[width=0.85\textwidth]{img/fig2_mapa_error.png}
    \end{center}
    \footnotesize{Error absolut |Original - Reconstruïda|. Distribució uniforme, sense artefactes.}
\end{frame}

\begin{frame}{Raspberry Pi 3B+ com a Simulador de CubeSat}
    \begin{columns}[T]
        \begin{column}{0.48\textwidth}
            \textbf{Per què Raspberry Pi?}
            \begin{itemize}
                \item ARM Cortex-A53 = mateixa arquitectura que Xilinx Zynq UltraScale+ (OBCs reals)
                \item 1 GB RAM simula restriccions reals
                \item microSD lenta → simula flash de vol
                \item Cost baix per experimentació
            \end{itemize}
        \end{column}
        \begin{column}{0.48\textwidth}
            \textbf{Validació:}
            \begin{itemize}
                \item NASA té guies per Raspberry Pi a l'espai
                \item Estudis IEEE confirmen Cortex-A53 per processament d'imatges a bord
                \item Codi portable a HW de vol real
            \end{itemize}
        \end{column}
    \end{columns}
\end{frame}

\end{document}