\documentclass[a4paper,11pt]{article}
\usepackage{graphicx} 
\usepackage[utf8]{inputenc}
\usepackage[T1]{fontenc}
\usepackage[catalan]{babel}
\usepackage{geometry}
\usepackage{hyperref}
\usepackage{csquotes}
\usepackage{url}
\usepackage{booktabs}
\usepackage{enumitem}
\usepackage{mathptmx} 
\usepackage{microtype} 



\geometry{margin=2.5cm}

% Configuración de hyperref
\hypersetup{
    colorlinks=true,
    linkcolor=blue,
    urlcolor=blue,
    citecolor=blue
}

\title{Implementació d'un compressor basat en xarxes neuronals\\per a ús en satèl·lit}
\author{Cristhian Omar Añez López \\ 1635157}
\date{}

\begin{document}

\maketitle

\section*{Introducció}
L'avenç tecnològic recent en camps com la intel·ligència artificial, l'aprenentatge automàtic, les comunicacions 5G i la robòtica ha revolucionat múltiples sectors \cite{zhu2017,bariah2020-ntn}. Un àmbit particularment afectat és l'observació de la Terra mitjançant satèl·lits, on l'adquisició d'imatges multiespectrals i hiperespectrals genera volums de dades sense precedents \cite{bioucas2013,enmap2015}.

El principal repte rau en la necessitat de comprimir eficientment aquestes imatges d'alta resolució espectral per superar les limitacions de capacitat d'emmagatzematge i transmissió inherents als satèl·lits, especialment en les missions de baix cost i CubeSats \cite{cubesat101,cds-cubesat,selva2012}. Els estàndards internacionals actuals, com els protocols CCSDS 121, 122 i 123 \cite{ccsds123}, empren algoritmes tradicionals de compressió que, malgrat la seva eficàcia, presenten limitacions computacionals creixents davant l'augment de la resolució espectral i espacial de les imatges modernes.

En aquest context, les tècniques de compressió basades en xarxes neuronals emergeixen com una alternativa prometedora. SORTENY (\emph{Spectral Orthogonal Transform Encoder}) \cite{sorteny2024} representa un avenç significatiu en aquest camp, implementant un sistema de compressió amb pèrdua que explota les correlacions espectrals i espacials mitjançant transformades apreses. Aquest enfoque ha demostrat capacitats superiors de compressió mantenint la qualitat de la informació espectral.

No obstant això, la implementació de SORTENY en entorns espacials reals presenta desafiaments considerables degut a les restriccions de potència, pes i capacitat computacional dels satèl·lits. Aquestes limitacions poden simular-se i estudiar-se utilitzant plataformes de desenvolupament de recursos limitats, com els ordinadors de placa única tipus Raspberry Pi, que ofereixen un entorn ideal per a l'optimització d'algoritmes destinats a aplicacions espacials.

\section*{Objectius}
	\subsection*{Objectiu general} Desenvolupar i validar una versió eficient del compressor SORTENY orientada a plataformes amb recursos limitats, minimitzant l'impacte de les restriccions de càlcul i memòria en el rendiment i la qualitat de la compressió.

	\subsection*{Objectius específics}
\begin{enumerate}[leftmargin=*,label=\alph*)]
  \item \textit{Implementació i disseny.} Implementar la inferència del compressor SORTENY en llenguatge C, amb una arquitectura clara i portable, adaptada a l'ordinador de placa utilitzat com a plataforma de simulació.
  \item \textit{Identificació i optimització de colls d'ampolla.} Localitzar de manera sistemàtica els colls d'ampolla de càlcul i memòria (transformacions, modulació/codificació, E/S) i aplicar-hi optimitzacions orientades a l'eficiència.
  \item \textit{Comparativa de rendiment i qualitat.} Mesurar les prestacions del sistema (ràtio de compressió, PSNR, latència i consum) i comparar-les amb solucions representatives basades en els estàndards CCSDS, mantenint el mateix entorn experimental.
  \item \textit{Proposta de millores.} Formular i validar millores de baix cost computacional que redueixin la càrrega de càlcul preservant la qualitat percebuda.
\end{enumerate}

\section*{Fases i metodologia}
El treball es divideix en cinc fases senzilles. L'objectiu és avançar pas a pas, comprovant cada fita abans de continuar.

\begin{enumerate}
\item \textbf{Revisió dels softwares de compressió i de les capacitats hardware}
\begin{itemize}
  \item Lectura i resum dels estàndards CCSDS i de SORTENY per entendre què resolen i en quines condicions funcionen millor.
  \item Visió general de què pot oferir el maquinari típic (CubeSats, ordinadors monoplaca) per fixar expectatives realistes.
\end{itemize}

\item \textbf{Implementació en C}
\begin{itemize}
  \item Traslladar el compressor SORTENY a C de forma incremental, prioritzant que compili i s'executi en local sense errors.
  \item Objectiu d'aquesta fase: obtenir una primera versió funcional que produeixi sortides coherents amb la línia base.
\end{itemize}

\item \textbf{Preparació de l'entorn de treball}
\begin{itemize}
  \item Posada a punt d'una \emph{Raspberry Pi} o entorn equivalent amb les eines necessàries.
  \item Objectiu d'aquesta fase: poder executar la versió en C a la placa i comprovar que tot arrenca correctament (\emph{prova de fum}).
\end{itemize}

\item \textbf{Avaluació experimental}
\begin{itemize}
  \item Preparar un petit conjunt d'imatges de prova i mesurar resultats (ràtio de compressió, qualitat i temps) de manera ordenada.
  \item Comparar amb una referència basada en CCSDS en les mateixes condicions i explicar què es guanya i què es perd amb cada canvi.
\end{itemize}

\item \textbf{Documentar per a propers treballs}
\begin{itemize}
  \item Elaborar un informe clar amb conclusions i deixar passos senzills per repetir les proves (com compilar, com executar i amb quines dades).
  \item Recollir idees de millora que han quedat fora i propostes concretes perquè una altra persona pugui continuar fàcilment quan calgui.
\end{itemize}
\end{enumerate}





\begin{thebibliography}{99}

\bibitem{zhu2017}
X.~X.~Zhu, D.~Tuia, L.~Mou, G.-S.~Xia, L.~Zhang, F.~Xu i J.~A.~Benediktsson,
``Deep Learning in Remote Sensing: A Comprehensive Review and List of Resources'',
\emph{IEEE Geoscience and Remote Sensing Magazine}, vol.~5, no.~4, pp.~8--36, 2017. doi:10.1109/MGRS.2017.2762307

\bibitem{bariah2020-ntn}
L.~Bariah, L.~A.~DaSilva, E.~Caldas, W.~Saad, M.~Debbah, H.~V.~Poor i C.~S.~Hong,
``A Prospective Look at Integrated Satellite-Terrestrial Networks for 6G and Beyond'',
\emph{IEEE Network}, vol.~35, no.~3, pp.~118--125, 2021. doi:10.1109/MNET.011.2000539

\bibitem{bioucas2013}
J.~M.~Bioucas-Dias, A.~Plaza, G.~Camps-Valls, P.~Scheunders, N.~M.~Nasrabadi i J.~Chanussot,
``Hyperspectral Remote Sensing Data Analysis and Future Challenges'',
\emph{IEEE Geoscience and Remote Sensing Magazine}, vol.~1, no.~2, pp.~6--36, 2013. doi:10.1109/MGRS.2013.2244672

\bibitem{enmap2015}
K.~Guanter, H.~Kaufmann, C.~Dobber, et al.,
``The EnMAP Spaceborne Imaging Spectroscopy Mission for Earth Observation'',
\emph{Remote Sensing}, vol.~7, no.~7, pp.~8830--8857, 2015. doi:10.3390/rs70708830

\bibitem{sorteny2024}
S.~Mijares, J.~Bartrina-Rapesta, M.~Hernández-Cabronero i J.~Serra-Sagristà,
``Learned Spectral and Spatial Transforms for Multispectral Remote Sensing Data Compression'',
\emph{IEEE Geoscience and Remote Sensing Letters}, vol. 22, no. 5001005, pp. 1-5, 2025.
10.1109/LGRS.2025.3554269.

\bibitem{ccsds123}
Consultative Committee for Space Data Systems (CCSDS),
\emph{Lossless Multispectral and Hyperspectral Image Compression (CCSDS 123.0-B-2)},
Washington, DC, USA, 2019.

\bibitem{cubesat101}
NASA, ``CubeSat 101: Basic Concepts and Processes for First-Time CubeSat Developers'', NASA, 2017. 
\url{https://www.nasa.gov/wp-content/uploads/2017/03/nasa_csli_cubesat_101_508.pdf}

\bibitem{cds-cubesat}
Cal Poly SLO, ``CubeSat Design Specification'', Rev.~14.1, 2014. 
\url{https://static1.squarespace.com/static/5418c831e4b0fa4ecac1bacd/t/62193b7fc9e72e0053f00910/1645820809779/CDS+REV14_1+2022-02-09.pdf}

\bibitem{selva2012}
A.~Selva i D.~Krejci, ``A survey and assessment of the capabilities of CubeSats for Earth observation'', 
\emph{Acta Astronautica}, vol.~74, pp.~50--68, 2012. doi:10.1016/j.actaastro.2011.12.014

\end{thebibliography}

\end{document}
